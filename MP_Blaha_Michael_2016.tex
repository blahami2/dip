% % % % % % % % % % % % %
% Example thesis (CTU FIT) based on CTUstyle 
% use: pdfcsplain <FIT-ZP.tex>
% write this document as UTF-8
% % % % % % % % % % % % %
% CTUstyle: see more at http://petr.olsak.net/ctustyle.html
% requirements: ctustyle.tex, ctulogo.pdf
% CTUstyle requires OPmac (opmac.tex)
% download OPmac from
% http://petr.olsak.net/opmac.html
% % % % % % % % % % % % %

% This example created by Ondrej Guth <ondrej.guth@fit.cvut.cz> 2014
% based on CTUstyle by Petr Olsak

% % % % % % % % % % % % % 
% The following not to be modified
% % % % % % % % % % % % % 
\input ctustyle
\faculty {F8}
% % % % % % % % % % % % % 
% Modify the following
% % % % % % % % % % % % %

% (thesis type {B,M,D})/(language {CZ,EN})
\worktype [M/EN]
\department {Katedra teoretické informatiky}
\title {Tablet infotainment system}
\author {Bc. Michael Bl\' aha}
\date {January 2016}
\supervisor{Ing Jan \v Sediv\' y, CSc.}
\abstractEN {This document is for testing purpose only.}
\abstractCZ {Tento dokument je pouze pro pot{\v r}eby testování.}
\declaration {Prohla\v suji, \v ze jsem p\v redlo\v zenou pr\'aci vypracoval(a) samostatn\v e a \v ze jsem uvedl(a) ve\v sker\'e pou\v zit\'e informa\v cn\'\i{} zdroje v~souladu s~Metodick\'ym pokynem o~etick\'e p\v r\'\i{}prav\v e vysoko\v skolsk\'ych z\'av\v ere\v cn\'ych prac\'\i{}.

		Beru na v\v edom\'\i{}, \v ze se na moji pr\'aci vztahuj\'\i{} pr\'ava a povinnosti vypl\'yvaj\'\i{}c\'\i{} ze z\'akona \v c.~121/2000~Sb., autorsk\'eho z\'akona, ve zn\v en\'\i{} pozd\v ej\v s\'\i{}ch p\v redpis{\r u}, zejm\'ena skute\v cnost, \v ze \v Cesk\'e vysok\'e u\v cen\'\i{} technick\'e v~Praze m\'a pr\'avo na uzav\v ren\'\i{} licen\v cn\'\i{} smlouvy o~u\v zit\'\i{} t\'eto pr\'ace jako \v skoln\'\i{}ho d\'\i{}la podle \S{}~60 odst.~1 autorsk\'eho z\'akona.}


\makefront



%------------------------------------------------------------------------ INTRODUCTION ------------------------------------------------------------------------%
\chap Introduction


\url{https://docs.google.com/document/d/1pGtlS5uY4PdKfHjf83dFGrajyVcea0tvzAnwXf61Cs8/edit}

generally progress: \url{https://docs.google.com/document/d/1CEWym7MphsCO0v3CXe_bTHOgFBgquBbPVbhk_pHZ5t8/edit}


%	------------------------------------------------------------------------ Project ----------------------------------------------------------------------%
\sec Project

See above

%		------------------------------------------------------------------------ Motivation ----------------------------------------------------------------------%
\secc Motivation

See above

%	------------------------------------------------------------------------ Assignment analysis ----------------------------------------------------------------------%
\sec Assignment analysis

\secc Review existing Android applications for in-car use

One of the key approches in research project is reviewing the existing progress in the given field. Reviewing existing applications helps understanding the topic, seeing the bigger picture, learning from mistakes of others and last but not least, getting general idea about competition.

\secc Review and analyse User Interface development methods for in-car infotainment applications

Cosindering the car environment, the user interface must deal with a lot of different problems than usual. This task should review existing User Interface development rules and apply them to the car environment, then analyse them and choose proper method for car-UI design process.

\secc Analyze the in-car OBD API and exported data

On-Board Diagnostics API is a standard API provided by modern cars for gathering various information from speed to engine temperature. This task focuses on understanding and gathering data from the OBD API.

\secc Design an application system architecture for accessing the OBD data and resources

Having the data from OBD and preparing an application for displaying them, designing proper architecture is required for everything to work well. The application has to gather data, while displaying them properly without unneccesary (FIX!) delay.

\secc Design a tablet User Interface for in-car use

After reviewing existing applications and UI development methods, the next goal is to create new User Interface for in-car use, while considering the constraints this environment puts on it.

\secc Design and implement in-car application offering the OBD data for Android tablet platform

With everything prepared and thought through, the application will be developed based on result from all the tasks accomplished so far. In this case, the Android platform will be used as explained later in the text.

\secc Perform UI and application testing and evaluate results

For best results the application must and will be tested. Both code and UI must be tested properly, using various testing approaches, such as unit tests or UI testing with reaul users in a car simulator.

%------------------------------------------------------------------------ ANALYSIS ------------------------------------------------------------------------%
\chap Analysis

see \url{application analysis https://docs.google.com/document/d/1QyOiMzV0ikcDhPY3P5MsRL_80cCGzjoGfCACGvc-UL0/edit}
priority list \url{https://docs.google.com/document/d/1juKYgUUDSI5CmfzjR4BsYSPHVYCGqrWuejgbhqzw7kI/edit}

%	------------------------------------------------------------------------ Existing applications ----------------------------------------------------------------------%
\sec Existing applications


\secc Applications

see \url{https://docs.google.com/document/d/1p_pSGTUHEojOyP7ICCDNVV7RW1vn8iN_KECipC4Y9tY/edit}
\url{http://www.makeuseof.com/tag/5-best-dashboard-car-mode-apps-android-compared/}

\secc Torque

Starting with an empty screen, lot of settings are required before using this application, since there is no default mode. Adding new views is easy and intuitive, but still very confusing. The add menu lacks hierarchy and everything is just sorted array of various options. There is no cancel button when popping the menu dialog.

This application can actually show almost anything OBD provides. It supports differents types of display, but it is hard to tell by their names. Responsiveness it not smooth at all and launching the application in horizontal mode confuses it, everything behaves like if it was in vertical mode.

\seccc Advantages

\begitems
* Lot of data from OBD available,
* various layout settings and themes,
* HUD mode.
\enditems

\seccc Disadvantages

\begitems
* One-level confusing menu without hierarchy,
* limited size options for displays (3 types),
* lacks default mode with predefined displays,
* hard to place displays, the grid does not work well,
* slow and laggy.
\enditems

% screenshot

\secc CarHome Ultra

This application starts with a pop-up tutorial for it's elementary functionality, telling the user about the speedmeter, compass, weather forecast and customizable dashboard for launching external applications. In default it offers Google Maps, Google Navigation and voice search. Adding another external application shortcut is done by tapping the tab. Also there are basic settings, which offer brightness mode (day, night, auto), theme and safety options.

It appears to be just a simple application offering speed, compass, weather and external application launcher. The new version also displays location (address) and a phone version is able to respond to text messages. It also supports text to speech (on touch).

\seccc Advantages

\begitems
* Simple UI, easy to understand,
* responsive, fluent,
* possible to change units (mile/km, etc.),
* lot of themes available,
* adjustable update rates,
* a lot of different settings.
\enditems

\seccc Disadvantages

\begitems
* Small buttons on small screens (fixed 6 buttons),
* even smaller setting buttons
* limited functionality
* tapping weather makes the app speak for every single tap, no matter if it already speaks (it can speak for hours after few taps).
\enditems

\secc Car Dashdroid

\secc Ultimate Car Dock

\secc Google AutoUI


%	------------------------------------------------------------------------ Platforms ----------------------------------------------------------------------%
\sec Platforms

Possibilities: Android, iOS, WP(?)
State requirements, properties, criteria

%	------------------------------------------------------------------------ Android platform ----------------------------------------------------------------------%
\sec Android platform


\secc Architecture


\secc Specifics


%	------------------------------------------------------------------------ GUI ----------------------------------------------------------------------%
\sec GUI


\secc Basic principles

MI-NUR           

\seccc Consistence

Dont make user learn things twice

\seccc Simplicity of usage

KIS principle

\seccc Shortness of learning curve

Easy to learn, critical environment

\secc Car UI differences

What else to consider?

\seccc Controls

Sizes, big enough to touch 

\seccc Visibility

Sizes, fonts big enough to see, data

\seccc Contrast

Colors, visibility at night/day

\seccc Responsivness

Responsive, must see whats going an as fast as possible


%	------------------------------------------------------------------------ Server ----------------------------------------------------------------------%
\sec Server


\secc Functionality


\secc Data storage


\secc Communication


%	------------------------------------------------------------------------ Development and support tools ----------------------------------------------------------------------%
\sec Development and support tools


\secc Development environment


\secc Quality Assurance tools


\secc Version system


\secc Test driven development


\secc Continuous integration



%------------------------------------------------------------------------ DESIGN ------------------------------------------------------------------------%
\chap Design


%	------------------------------------------------------------------------ Application architecture ----------------------------------------------------------------------%
\sec Application architecture


\secc Extensibility


\secc Modularity


\secc Adaptability


\secc AutoUI preparation


\secc Platform limitations


%	------------------------------------------------------------------------ GUI ----------------------------------------------------------------------%
\sec GUI


\secc Basic elements

Basic idea

\secc UI drafts

Describe the process, phases, analyse and compare advantages, disadvantages, thoughts

%------------------------------------------------------------------------ REALIZATION ------------------------------------------------------------------------%
\chap Realization


%	------------------------------------------------------------------------ Preparation ----------------------------------------------------------------------%
\sec Preparation


\secc Environment


\secc Versioning


\secc Testing


\secc Scripting


%	------------------------------------------------------------------------ Core ----------------------------------------------------------------------%
\sec Core


\secc Core


\secc Data storage


\secc Communication


\secc Optimization


%	------------------------------------------------------------------------ Modularity ----------------------------------------------------------------------%
\sec Modularity


\secc Requirements


\secc Integration


%	------------------------------------------------------------------------ GUI ----------------------------------------------------------------------%
\sec GUI

GUI implementation based on the design! Implementing modules, color, responsive effects

\secc Common elements

Hierarchical model, effects, submenus

\secc Multiple designs

Limited set of module types

%------------------------------------------------------------------------ TESTING ------------------------------------------------------------------------%
\chap Testing

Brag about TDD, CI and Simulator!

%	------------------------------------------------------------------------ Implementation ----------------------------------------------------------------------%
\sec Code

Describe testing code, common testing (look\&see, etc.)

\secc Unit testing

Unit testing on android, mention Test driven development, continuous integration, automatic tests, consider giving an example

\secc Integration testing

Instrumentation? Describe TDD, CI, automation

\secc System testing

Server testing, consider removing

\secc Qualification testing

Testing with users - consider section on its own - testing application as a whole thing

%	------------------------------------------------------------------------ GUI ----------------------------------------------------------------------%
\sec GUI


\secc Heuristic testing

Introduction, description

\seccc Evaluation

\url{https://docs.google.com/document/d/1LAPqmYqe5LBE6vqWpi-rRYjHY1-zVPCzkFP2Gvh5i-Q/edit}

\seccc Conclusion

Did not have time to fix

\secc Testing with users


\seccc Usability testing

Testing the application as a regular application. Is it understandable? Is it easy to control, to see data, to understand, to comprehend, to learn?


\seccc Simulator

Describe the car simulator in Albertov.
DO NOT FORGET TO THANK THE DEPARTMENT OF DRIVING SMTHING, CVUT FD


\seccc Preparations

Selecting the world models and preparing them for testing, installing EyeTracker cameras, installing WebCamera, preparing data gathering, designing scenarios

\seccc Course

The testing itself, describing participants

\seccc Evaluation

Evaluating results

%	------------------------------------------------------------------------ Summary ----------------------------------------------------------------------%
\sec Summary


%------------------------------------------------------------------------ CONCLUSION ------------------------------------------------------------------------%
\chap Conclusion


%	------------------------------------------------------------------------ Assignment completion ----------------------------------------------------------------------%
\sec Assignment completion


%	------------------------------------------------------------------------ Project life cycle ----------------------------------------------------------------------%
\sec Project life cycle


\secc Present


\secc Future


%	------------------------------------------------------------------------ Summary ----------------------------------------------------------------------%
\sec Summary



\bye
