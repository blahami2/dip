% % % % % % % % % % % % %
% Example thesis (CTU FIT) based on CTUstyle 
% use: pdfcsplain <FIT-ZP.tex>
% write this document as UTF-8
% % % % % % % % % % % % %
% CTUstyle: see more at http://petr.olsak.net/ctustyle.html
% requirements: ctustyle.tex, ctulogo.pdf
% CTUstyle requires OPmac (opmac.tex)
% download OPmac from
% http://petr.olsak.net/opmac.html
% % % % % % % % % % % % %

% This example created by Ondrej Guth <ondrej.guth@fit.cvut.cz> 2014
% based on CTUstyle by Petr Olsak

% % % % % % % % % % % % % 
% The following not to be modified
% % % % % % % % % % % % % 
\input ctustyle
\faculty {F8}
% % % % % % % % % % % % % 
% Modify the following
% % % % % % % % % % % % %

% (thesis type {B,M,D})/(language {CZ,EN})
\worktype [M/EN]
\department {Katedra teoretické informatiky}
\title {Tablet infotainment system}
\author {Bc. Michael Bl\' aha}
\date {January 2016}
\supervisor{Ing Jan \v Sediv\' y, CSc.}
\abstractEN {This document is for testing purpose only.}
\abstractCZ {Tento dokument je pouze pro pot{\v r}eby testování.}
\declaration {Prohla\v suji, \v ze jsem p\v redlo\v zenou pr\'aci vypracoval(a) samostatn\v e a \v ze jsem uvedl(a) ve\v sker\'e pou\v zit\'e informa\v cn\'\i{} zdroje v~souladu s~Metodick\'ym pokynem o~etick\'e p\v r\'\i{}prav\v e vysoko\v skolsk\'ych z\'av\v ere\v cn\'ych prac\'\i{}.

		Beru na v\v edom\'\i{}, \v ze se na moji pr\'aci vztahuj\'\i{} pr\'ava a povinnosti vypl\'yvaj\'\i{}c\'\i{} ze z\'akona \v c.~121/2000~Sb., autorsk\'eho z\'akona, ve zn\v en\'\i{} pozd\v ej\v s\'\i{}ch p\v redpis{\r u}, zejm\'ena skute\v cnost, \v ze \v Cesk\'e vysok\'e u\v cen\'\i{} technick\'e v~Praze m\'a pr\'avo na uzav\v ren\'\i{} licen\v cn\'\i{} smlouvy o~u\v zit\'\i{} t\'eto pr\'ace jako \v skoln\'\i{}ho d\'\i{}la podle \S{}~60 odst.~1 autorsk\'eho z\'akona.}


\picw=.8\hsize

\makefront



%------------------------------------------------------------------------ INTRODUCTION ------------------------------------------------------------------------%
\chap Introduction TODO

sources:
\begitems
* \url{https://docs.google.com/document/d/1pGtlS5uY4PdKfHjf83dFGrajyVcea0tvzAnwXf61Cs8/edit}
* generally progress: \url{https://docs.google.com/document/d/1CEWym7MphsCO0v3CXe_bTHOgFBgquBbPVbhk_pHZ5t8/edit}
\enditems

%	------------------------------------------------------------------------ Project ----------------------------------------------------------------------%
\sec Project TODO

See above

%		------------------------------------------------------------------------ Motivation ----------------------------------------------------------------------%
\secc Motivation TODO

See above

%	------------------------------------------------------------------------ Assignment analysis ----------------------------------------------------------------------%
\sec Assignment analysis

\secc Assignment tasks

\seccc Review existing Android applications for in-car use

One of the key approches in research project is reviewing the existing progress in the given field. Reviewing existing applications helps understanding the topic, seeing the bigger picture, learning from mistakes of others and last but not least, getting general idea about competition.

\seccc Review and analyse User Interface development methods for in-car infotainment applications

Cosindering the car environment, the user interface must deal with a lot of different problems than usual. This task should review existing User Interface development rules and apply them to the car environment, then analyse them and choose proper method for car-UI design process.

\seccc Analyze the in-car OBD API and exported data

On-Board Diagnostics API is a standard API provided by modern cars for gathering various information from speed to engine temperature. This task focuses on understanding and gathering data from the OBD API.

\seccc Design an application system architecture for accessing the OBD data and resources

Having the data from OBD and preparing an application for displaying them, designing proper architecture is required for everything to work well. The application has to gather data, while displaying them properly without unneccesary (TODO FIX!) delay.

\seccc Design a tablet User Interface for in-car use

After reviewing existing applications and UI development methods, the next goal is to create new User Interface for in-car use, while considering the constraints this environment puts on it.

\seccc Design and implement in-car application offering the OBD data for Android tablet platform

With everything prepared and thought through, the application will be developed based on result from all the tasks accomplished so far. In this case, the Android platform will be used as explained later in the text.

\seccc Perform UI and application testing and evaluate results

For best results the application must and will be tested. Both code and UI must be tested properly, using various testing approaches, such as unit tests or UI testing with reaul users in a car simulator.

%------------------------------------------------------------------------ ANALYSIS ------------------------------------------------------------------------%
\chap Analysis TODO

sources:
\begitems
* application analysis \url{https://docs.google.com/document/d/1QyOiMzV0ikcDhPY3P5MsRL_80cCGzjoGfCACGvc-UL0/edit}
* priority list \url{https://docs.google.com/document/d/1juKYgUUDSI5CmfzjR4BsYSPHVYCGqrWuejgbhqzw7kI/edit}
\enditems

%	------------------------------------------------------------------------ Existing applications ----------------------------------------------------------------------%
\sec Existing applications


\secc Applications TODO

sources:
\begitems
* \url{https://docs.google.com/document/d/1p_pSGTUHEojOyP7ICCDNVV7RW1vn8iN_KECipC4Y9tY/edit}
* \url{http://www.makeuseof.com/tag/5-best-dashboard-car-mode-apps-android-compared/}
\enditems

\secc Torque

Starting with an empty screen, lot of settings are required before using this application, since there is no default mode. Adding new views is easy and intuitive, but still very confusing. The add menu lacks hierarchy and everything is just sorted array of various options. There is no cancel button when popping the menu dialog.

This application can actually show almost anything OBD provides. It supports differents types of display, but it is hard to tell by their names. Responsiveness it not smooth at all and launching the application in horizontal mode confuses it, everything behaves like if it was in vertical mode.


\medskip
\centerline{\inspic images/analysis_app_torque.png }\nobreak\medskip
\caption/f Screenshot from Torque

\seccc Advantages

\begitems
* Lot of data from OBD available,
* various layout settings and themes,
* HUD mode.
\enditems

\seccc Disadvantages

\begitems
* One-level confusing menu without hierarchy,
* limited size options for displays (3 types),
* lacks default mode with predefined displays,
* hard to place displays, the grid does not work well,
* slow and laggy.
\enditems

% screenshot

\secc CarHome Ultra

This application starts with a pop-up tutorial for it's elementary functionality, telling the user about the speedmeter, compass, weather forecast and customizable dashboard for launching external applications. In default it offers Google Maps, Google Navigation and voice search. Adding another external application shortcut is done by tapping the tab. Also there are basic settings, which offer brightness mode (day, night, auto), theme and safety options.

It appears to be just a simple application offering speed, compass, weather and external application launcher. The new version also displays location (address) and a phone version is able to respond to text messages. It also supports text to speech (on touch).

\medskip
\centerline{\inspic images/analysis_app_carhomeultra.png }\nobreak\medskip
\caption/f Screenshot from CarHome Ultra

\seccc Advantages

\begitems
* Simple UI, easy to understand,
* responsive, fluent,
* possible to change units (mile/km, etc.),
* lot of themes available,
* adjustable update rates,
* a lot of different settings.
\enditems

\seccc Disadvantages

\begitems
* Small buttons on small screens (fixed 6 buttons),
* even smaller setting buttons
* limited functionality
* tapping weather makes the app speak for every single tap, no matter if it already speaks (it can speak for hours after few taps).
\enditems

\secc Car Dashdroid

After a long loading the main window appears. It has three screens, which change by swiping right or left. The left screen contains dial keyboard, the right screen contains customizable cards (for external application shortcuts or built-in tools) and the main screen consists of weather, speed and shortcuts to contacts, music, navigation and voice command.

It also provides settings for bluetooth, brightness, screen rotation, fullscreen, day/night mode and application settings, where other options can be set, such as home adress, theme, units.

\medskip
\centerline{\inspic images/analysis_app_cardashdroid.png }\nobreak\medskip
\caption/f Screenshot from Car Dashdroid

\seccc Advantages

\begitems
* Simple UI, easy to understand,
* responsive, fluent,
* possible to change units (mile/km, etc.),
* able to read incoming SMS using TTS.
\enditems

\seccc Disadvantages

\begitems
* Very limited functionality
* not optimized for tablet,
* distractive commercial ads in free version.
\enditems

\secc Ultimate Car Dock

While the design is very similar to CarHome Ultra, this application offers fewer displays on a single screen. There are five screens, each one consists of six cards. Every card can change into shortcut or a build-in application. The Ultimate Car Dock has only few built-in applications: music player, voice command, speed, weather, messages and calls. It also supports shortcuts to other external applications.

\medskip
\centerline{\inspic images/analysis_app_ultimatecardock.png }\nobreak\medskip
\caption/f Screenshot from Ultimate Car Dock

\seccc Advantages

\begitems
* Simple UI, easy to understand,
* responsive, fluent,
* possible to change units (mile/km, etc.),
* able to read various incoming notifications using TTS (Gmail, WhatsApp, etc.),
* predefined SMS responses (selectable when a message comes),
* direct calls and messages (shortcut to call/message a certain person).
\enditems

\seccc Disadvantages

\begitems
* Limited functionality
* not optimized for tablet,
* small font.
\enditems

\secc Conclusion

Except by Torque, which focuses mainly (and only) on OBD, all the applications are very similar to each other. They have similar design and functionality -- mostly weather, speed provided by GPS, voice command and shortcuts for external applications.

\seccc Suggestions

\begitems
* OBD data,
* shortcuts to other applications,
* adjustable cards,
* built-in cards (weather, speed, voice command, etc.),
* simple grid UI,
* possibility to change displayed units,
* responsive and fluent,
* day and night theme,
* predefined message and call responses,
* TTS for incoming notifications.
\enditems

\seccc Possible issues to avoid

\begitems
* Responsiveness,
* limited functionality,
* small and hardly visible font,
* distractive ads.
\enditems

\secc Android Auto

sources:
\begitems
* \url{http://developer.android.com/design/auto/index.html}
* \url{https://www.google.com/design/spec-auto/designing-for-android-auto/designing-for-cars.html}
\enditems

Recently, Google presented new application model for information delivery while driving. It is called Android Auto, it provides a standardized user interface and user interaction model for Android devices. Focusing on minimazing the driver distraction, it presents a few options to interact with user. It supports three application types:
\begitems
* System overview
* Audio applications
* Messaging applications
\enditems

\seccc System overview

System overview is supposed to be a home screen for Android Auto application. It presents both current and past notifications. The amount of notifications is limited based on screen size. Every notification consists of an intent icon, text and image, while following certain sizing rules. Every such notification can be expanded on the spot or another subapplication can be launched.

\medskip
\centerline{\inspic images/analysis_aa_overview.png }\nobreak\medskip
\caption/f Android Auto Home screen

\seccc Audio applications

Audio applications in Android Auto have a simple template structure. It consists of a main consumption view, a drawer and a queue screen. The main consumption view displays a few control elements and a cover background. The drawer is a simple list and provides access to favorite and popular content. Finally the queue screen displays a list of pending content, for example songs in a queue.

\medskip
\centerline{\inspic images/analysis_aa_audio.png }\nobreak\medskip
\caption/f Android Auto audio application

\seccc Messaging applications

Focusing on minimizing the cognitive load, messaging concept in Android Auto focuses on voice control over looking and typing. It allows reading the message outloud and responding with a set of predefined voice commands as well as dictating a whole message using built-in speech recognition.

\medskip
\centerline{\inspic images/analysis_aa_message.png }\nobreak\medskip
\caption/f Android Auto conversational flow

\seccc Conclusion

It seems to be a good sign that even Google is interested in this area and performs such a research. Every Android application can be designed for Android Auto and use it's simplified user interface, allowing the developer to focus on other issues than in-car user interaction. However, the functionality is currently very limited. Hopefully there will be further progress as soon as possible.

%	------------------------------------------------------------------------ Platforms ----------------------------------------------------------------------%
\sec Platforms

The chosen platform heavily influences the piece of market an application can reach. Therefore, only platforms with solid market share are considered. Another criteria is the simplicity of development, which influences the time and effort put into an application before it can be released. This is especially important for quickly finding out the sale potential of an application.
Following the first rule mentioned above and based on tablet sales in past years (sources: \url{http://techcrunch.com/2014/03/03/gartner-195m-tablets-sold-in-2013-android-grabs-top-spot-from-ipad-with-62-share/}), the only viable options for an application are platforms Android, iOS and Windows.

\secc Android

In 2013, the Android platform had 61.9 \% market share, making it the most used platform in the world. Targeting the Android platform would create large base of potential customers.

The development language for Android is Java, commonly known object-oriented programming language with solid developer base. Therefore it is easy to find developers as well as answers to variety of programming related issues, making the development much easier.

\secc iOS

With 36 \% market share in 2013, iOS is the second most popular tablet platform. Considering a typical iOS user, who is willing to pay for quality, iOS could be a good choice for an application in context of potential customers.

However, the development language called Swift is somewhat new in the world, which brings a lot of possible difficulties. Searching for answers while developing in this technology might prove too troublesome.

\secc Windows

With only 2.1 \% market share in 2014, the Windows platform does not seem to be a valid choise for given criteria. Having thirty times lower customer base than Android, it goes into the nice-to-have section when it comes to multi-platform applications.

\secc Conclusion

Fulfiling the requirements for customer base as well as simplicity of development, the Android platform seems to be the best choice available at the time of writing this. As such, it will be analyzed more thoroughly later in this chapter.

%	------------------------------------------------------------------------ Android platform ----------------------------------------------------------------------%
\sec Android platform TODO

There are some platform aspects to be considered when developing for Android-based tablet device. First is the problem with most of the tablet devices -- the device performance. While the hardware is continuously evolving, one must consider older devices as well as growing requirements for graphics presentation. The second possible issue is the Android architecture, which influences the application inner communication.

\secc Performance

Nearly with every new version of Android, new presentation effects are prepared for developers to use. While it is not mandatory, it is still adviced to hold the platform standards as the market demands it. The application must have a good look and feel in order to attract attention. This must be considered when building the application, because the environment demands fluent responses.

\secc Architecture

The main building block in Android application is an activity. The activity is an idependent component of application, a hybrid between controller and view in MVC architecture. It contains a single screen (which contains a single layout), it has it's own independent data. An application usually consists of multiple loosely coupled activities. Those activities are held in an activity stack, where they are preserved to be used later without need to create them all over again. However, if the system needs memory, it clears the stack from the bottom (least recently used activities).
For communication between activities there are so called Intents. An intent is the main concept of communication between two components. A component can be for example an activity or a service. Intent can contain simple data, such as primitive or serialized data.

%------------------------------------------------------------------------ GUI ----------------------------------------------------------------------%
\sec GUI TODO



\secc Basic principles

As the main tool of communication between an application and it's user, user interface must follow one basic rule -- the user goes first. UI is about the user, he must have a good feeling when using the application. He must understand what to do and how to do it. Therefore there are four rules a proper UI must obey:
\begitems
* {\bf Clear} -- it must be obvious what and where the user can control,
* {\bf Effective} -- minimizing the required user interactions for certain (requested) thing to happen,
* {\bf Foolproof} - avoiding the errors before they happen,
* {\bf Pleasant} - no stress when working with the UI, pleasant colors, contrast, good readability.
\enditems

Those rules might seem too shallow. That is why there are certain subgoals which are more specific, helping to achieve the main four goals. Those subgoals are the following seven:
\begitems
* {\bf Minimality} -- removing everything that can be removed without losing the requested information value,
* {\bf Responsiveness} -- giving the user a proper feedback, so that he knows something is happening,
* {\bf Forgiveness} -- letting the user make mistakes, allowing him to fix them, for example undo button or prompt message,
* {\bf Familiarity} -- using familiar, commonly used metaphors, icons, procedures,
* {\bf Consistency} -- using consistent visual and interaction language,
* {\bf Integration} -- using platform specific elements and rules
* {\bf Simplicity} - allowing user to quickly learn how to use the UI
\enditems

sources:
\begitems
* MI-NUR \url{https://edux.fit.cvut.cz/courses/MI-NUR/lectures/start}
* Designing for indash automotive \url{http://revinity.com/?p=128 }
* UX design stackexchange \url{http://ux.stackexchange.com/questions/51968/what-ux-guidelines-should-one-keep-in-mind-when-designing-the-gui-for-a-automobi}         0
\enditems

\secc UI in a car environment

When developing user interface for car, it adds a certain responsibility. The need for safety while using the UI becomes the main priority. Because of that, some aspects are more important than other. The most important are described later in this section.

\seccc Minimality

For minimizing the cognitive load, the must be as little information as possible at a certain time. The user must see what he wants to see on first sight without seeking the answer for too long. When minimizing the information displayed there is no confusion, minimizing the glance time.

\seccc Consistence

Supporting usability and shortness of learning curve, consistency allows user to remember one procedure and apply it successfuly in different sections of UI. It allows user to learn things just once.

\seccc Readability

Good readability is one of the condition for application to be pleasant to use. In case of car environment, however, the readability of information is not just pleasant, but critical. Allowing the user to see the information he needs to see in the shortest time possible is fatal when it comes to driving. Therefore the font has to be large enough for every driver to see.

\seccc Controls

When it comes to controlling an application in an environment such as car, it is required to consider certain aspects not present in other environments. The moving car prevents user from being precise when it comes to touch. Therefore, the controls must be large enough to be reliably selectable. 

\seccc Colors

While in other environment the user can usually control the device brightness, it is not as easy task while driving. Furthermore, blinding the user with too much light might be fatal. Therefore proper colors must be used. For example, dominance of white color might be visible well in the daylight, but might blind the user at the night time. Also, proper color contrast must be considered for good visibility and readability.

\seccc Responsiveness

Responsiveness is an important factor when it comes to pleasure of using an application, but when it comes to using it in a car, it becomes extremely important for safety as well. When the application is responsive, the user does not have to check the screen for progress so often or worse, wait for the progress looking at it continuously.


%	------------------------------------------------------------------------ Server ----------------------------------------------------------------------%
%\sec Server
%
%
%\secc Functionality
%
%
%\secc Data storage
%
%
%\secc Communication
%
%
%	------------------------------------------------------------------------ Development and support tools ----------------------------------------------------------------------%
\sec Development and support tools TODO


\secc Development environment

While using text editor and command line is an option, for speed of development only Integrated Development Environments are considered. In the time of writing this text, there were two possibilities for Android development -- Eclipse and AndroidStudio.

\seccc Eclipse

Based on research by (\url{http://zeroturnaround.com/rebellabs/java-tools-and-technologies-landscape-for-2014/6/}), the Eclipse IDE is the most often used Java IDE. That is probably the reason why Google inc. suggested this IDE for Android development in early phase. However, Eclipse has lost Android development support in late 2014 (\url{http://www.zdnet.com/article/google-releases-android-studio-kills-off-eclipse-adt-plugin/}).

\seccc Android Studio

Released in 2014, Android Studio became the main platform for Android development. It is based on IntelliJ IDE and supported by Google. For that reason, it is an obvious choice for new applications to be developed in Android Studio.


\secc Version control system

In a software development process, version is very important. Being able to go back to working version, or to develop new features while the main version is still working is priceless. Currently there are three main VCS worth considering (\url{http://www.sitepoint.com/version-control-software-2014-what-options/}). 

\seccc Subversion

Based on CVS, Subversion has a single repository where all the data are stored. This simplifies the backup of a whole project, because all the data are located in one place. This, however, creates possible threat of data loss when the central repository gets destroyed without backup.

Because of the central repository, Subversion allows read and write access controls for a every single location and have them enforced across the entire project, which can come in handy when developing in a large community, but it is usually not required when developing in a small team.


\seccc Mercurial TODO

\seccc Git TODO

\secc Test driven development

Being one of the main development aproaches in the last decade, test driven development helps to develop the application quickly and fluently. The main idea of TDD is to create automatic tests before the actual application code. While this enforces the developer to think twice when creating tests, which makes him think about what he actually wants to achieve, it also helps against random errors in code. Having the application tested with every build also supports continuous integration, which is described later in this text.

\secc Continuous integration

"Continuous Integration is a software development practice where members of a team integrate their work frequently, usually each person integrates at least daily - leading to multiple integrations per day. Each integration is verified by an automated build (including test) to detect integration errors as quickly as possible." (Martin Fowler, \url{http://www.martinfowler.com/articles/continuousIntegration.html})

Continuous integration supports rapid application development while giving the much needed feedback, so that the developer can see and adjust the direction, which the application development takes. 

For simple CI practice integration, there are several online services. For this particular application development, Travis system was chosen (TODO link \url{https://travis-ci.org/}). While offering usual CI functionality, it also integrates easily with GitHub and Gradle build system. 

%	------------------------------------------------------------------------ On-Board Diagnostics ----------------------------------------------------------------------%
\sec On-Board Diagnostics TODO

\secc Overview

On-Board Diagnostics states for a self-diagnostic equipment requirements for automotive vehicles. The modern implementations offer standardized communication port to provide real-time data as well as diagnostic trouble codes. The full specifications are explained in ISO 15031 (TODO fix). Some of them will be discussed later in the text.

Currently there are two versions of OBD. The first one (OBD I) provides only diagnostic trouble codes. The second one (OBD II) adds real-time vehicle data. The third version (OBD III) is currently being developed. It should support so called "remote OBD", which would broadcast the data to other vehicles, which could prevent collisions by warning the drivers when something bad happens suddenly. 

\secc API TODO



\secc Data TODO


%------------------------------------------------------------------------ DESIGN ------------------------------------------------------------------------%
\chap Design TODO


%	------------------------------------------------------------------------ Application architecture ----------------------------------------------------------------------%
\sec Application architecture TODO

Designing proper application architecture is one of the main and most challanging tasks in development process. Changing the architecture in to future proves to be one of the most expensive changes as for manhours (TODO link Code Complete?). Application architecture influences data flow, communication amongst components and overall application performance, as well as extensibility and possibility to change or add features in the future. While Android application architecture enforces certain components and platform features to be used, the is still space for diversity.

\secc Platform limitations

As mentioned in platform analysis (TODO link), the typical Android application consits of multiple Activities, which communicate with each other using Intents. While this approach supports the loosely coupled concept, it makes certain inter-cooperation rather difficult.  Sharing an object between activities usually means serializing the object or saving it to the database, which means deserializing or loading from database later. When striving for excellent performance, this can emerge into a real problem. As the Android platform does not allow database IO operations on the main presentation thread, it requires background thread with callbacks to the main one and screen revalidation when such callback occurs. It is critical to avoid such delays as much as possible, when comes to car environment where rapid reactions are required.

\secc Extensibility

With the current rapid application development there is a need to be able to adjust based on market requirements. While creating a new application with every feature is a possibility, it is certainly better to be able to add new features to the old application so that is actually never becomes old. Extensibility is one of the main requirements for many reasons. When it comes to CarDashboard application (TODO name?), new features are planned to be added based on user feedback. Therefore the architecture must be prepared to be easily extendable.

The main approach to achieve proper extensibility should be to write a clean code, which can prove to be a good idea when considering nearly every part of implementation process. Also the modularity concept is very usefull when it comes to extensibility and it will be discussed in the next section.

\secc Modularity TODO

Modularity concept allows application to contain certain modules, each offering certain functionality based on some predefined requirements. 

\secc Adaptability TODO


\secc AutoUI preparation TODO




%	------------------------------------------------------------------------ GUI ----------------------------------------------------------------------%
\sec GUI TODO


\secc Basic elements TODO

Basic idea

\secc UI drafts TODO

Describe the process, phases, analyse and compare advantages, disadvantages, thoughts

%------------------------------------------------------------------------ REALIZATION ------------------------------------------------------------------------%
\chap Realization TODO


%	------------------------------------------------------------------------ Preparation ----------------------------------------------------------------------%
\sec Preparation TODO


\secc Environment TODO


\secc Versioning TODO


\secc Testing TODO


\secc Scripting TODO


%	------------------------------------------------------------------------ Core ----------------------------------------------------------------------%
\sec Core TODO


\secc Core TODO


\secc Data storage TODO


\secc Communication TODO


\secc Optimization TODO


%	------------------------------------------------------------------------ Modularity ----------------------------------------------------------------------%
\sec Modularity TODO


\secc Requirements TODO


\secc Integration TODO


%	------------------------------------------------------------------------ GUI ----------------------------------------------------------------------%
\sec GUI TODO

GUI implementation based on the design! Implementing modules, color, responsive effects

\secc Common elements TODO

Hierarchical model, effects, submenus

\secc Multiple designs TODO

Limited set of module types

%------------------------------------------------------------------------ TESTING ------------------------------------------------------------------------%
\chap Testing TODO

Brag about TDD, CI and Simulator!

%	------------------------------------------------------------------------ Implementation ----------------------------------------------------------------------%
\sec Code TODO

Describe testing code, common testing (look\&see, etc.)

\secc Unit testing TODO

Unit testing on android, mention Test driven development, continuous integration, automatic tests, consider giving an example

\secc Integration testing TODO

Instrumentation? Describe TDD, CI, automation

\secc System testing TODO

Server testing, consider removing

\secc Qualification testing TODO

Testing with users - consider section on its own - testing application as a whole thing

%	------------------------------------------------------------------------ GUI ----------------------------------------------------------------------%
\sec GUI TODO


\secc Heuristic testing TODO

Introduction, description

\seccc Evaluation TODO

sources:
\begitems
* \url{https://docs.google.com/document/d/1LAPqmYqe5LBE6vqWpi-rRYjHY1-zVPCzkFP2Gvh5i-Q/edit}
\enditems

\seccc Conclusion TODO

Did not have time to fix

\secc Testing with users TODO


\seccc Usability testing TODO

Testing the application as a regular application. Is it understandable? Is it easy to control, to see data, to understand, to comprehend, to learn?


\seccc Simulator TODO

Describe the car simulator in Albertov.
DO NOT FORGET TO THANK THE DEPARTMENT OF DRIVING SMTHING, CVUT FD


\seccc Preparations TODO

Selecting the world models and preparing them for testing, installing EyeTracker cameras, installing WebCamera, preparing data gathering, designing scenarios

\seccc Course TODO

The testing itself, describing participants

\seccc Evaluation TODO

Evaluating results

%	------------------------------------------------------------------------ Summary ----------------------------------------------------------------------%
\sec Summary TODO


%------------------------------------------------------------------------ CONCLUSION ------------------------------------------------------------------------%
\chap Conclusion TODO


%	------------------------------------------------------------------------ Assignment completion ----------------------------------------------------------------------%
\sec Assignment completion TODO


%	------------------------------------------------------------------------ Project life cycle ----------------------------------------------------------------------%
\sec Project life cycle TODO


\secc Present TODO


\secc Future TODO


%	------------------------------------------------------------------------ Summary ----------------------------------------------------------------------%
\sec Summary TODO



\bye
